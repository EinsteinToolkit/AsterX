% *======================================================================*
%  Cactus Thorn template for ThornGuide documentation
%  Author: Ian Kelley
%  Date: Sun Jun 02, 2002
%  $Header$
%
%  Thorn documentation in the latex file doc/documentation.tex
%  will be included in ThornGuides built with the Cactus make system.
%  The scripts employed by the make system automatically include
%  pages about variables, parameters and scheduling parsed from the
%  relevant thorn CCL files.
%
%  This template contains guidelines which help to assure that your
%  documentation will be correctly added to ThornGuides. More
%  information is available in the Cactus UsersGuide.
%
%  Guidelines:
%   - Do not change anything before the line
%       % START CACTUS THORNGUIDE",
%     except for filling in the title, author, date, etc. fields.
%        - Each of these fields should only be on ONE line.
%        - Author names should be separated with a \\ or a comma.
%   - You can define your own macros, but they must appear after
%     the START CACTUS THORNGUIDE line, and must not redefine standard
%     latex commands.
%   - To avoid name clashes with other thorns, 'labels', 'citations',
%     'references', and 'image' names should conform to the following
%     convention:
%       ARRANGEMENT_THORN_LABEL
%     For example, an image wave.eps in the arrangement CactusWave and
%     thorn WaveToyC should be renamed to CactusWave_WaveToyC_wave.eps
%   - Graphics should only be included using the graphicx package.
%     More specifically, with the "\includegraphics" command.  Do
%     not specify any graphic file extensions in your .tex file. This
%     will allow us to create a PDF version of the ThornGuide
%     via pdflatex.
%   - References should be included with the latex "\bibitem" command.
%   - Use \begin{abstract}...\end{abstract} instead of \abstract{...}
%   - Do not use \appendix, instead include any appendices you need as
%     standard sections.
%   - For the benefit of our Perl scripts, and for future extensions,
%     please use simple latex.
%
% *======================================================================*
%
% Example of including a graphic image:
%    \begin{figure}[ht]
% 	\begin{center}
%    	   \includegraphics[width=6cm]{MyArrangement_MyThorn_MyFigure}
% 	\end{center}
% 	\caption{Illustration of this and that}
% 	\label{MyArrangement_MyThorn_MyLabel}
%    \end{figure}
%
% Example of using a label:
%   \label{MyArrangement_MyThorn_MyLabel}
%
% Example of a citation:
%    \cite{MyArrangement_MyThorn_Author99}
%
% Example of including a reference
%   \bibitem{MyArrangement_MyThorn_Author99}
%   {J. Author, {\em The Title of the Book, Journal, or periodical}, 1 (1999),
%   1--16. {\tt http://www.nowhere.com/}}
%
% *======================================================================*

% If you are using CVS use this line to give version information
% $Header$

\documentclass{article}

% Use the Cactus ThornGuide style file
% (Automatically used from Cactus distribution, if you have a
%  thorn without the Cactus Flesh download this from the Cactus
%  homepage at www.cactuscode.org)
\usepackage{../../../../doc/latex/cactus}

\begin{document}

% The author of the documentation
\author{Samuel Tootle \textless tootle@itp.uni-frankfurt.de\textgreater}

% The title of the document (not necessarily the name of the Thorn)
\title{KadathThorn}

% the date your document was last changed, if your document is in CVS,
% please use:
%    \date{$ $Date$ $}
\date{February 19, 2023}

\maketitle

% Do not delete next line
% START CACTUS THORNGUIDE

% Add all definitions used in this documentation here
%   \def\mydef etc
\newcommand{\hk}{\texttt{HOME\_KADATH}}
% Add an abstract for this thorn's documentation
\begin{abstract}
This thorn provides the base interface to the
Frankfurt University/KADATH (FUKA) library. For details on FUKA and
the initial data import, see the KadathImporter thorn documentation.
\end{abstract}

% The following sections are suggestive only.
% Remove them or add your own.

\section{Introduction}
Importing FUKA initial data currently requires linking the EinsteinToolkit executable to the built
FUKA library.  This thorn handles linking to an existing library or building one from scratch.

\section{Linking}
Linking FUKA to the ETK is done in two ways, linking to an existing FUKA library or building one 
during the ETK build process and linking after.

\subsection*{Existing library}
To build using an existing library, one can set the \texttt{KADATH\_DIR} variable in the ETK build
config file to the base directory of the cloned FUKA distribution.  

Alternatively, during the FUKA installation 
procedure, the base directory must be included in the \hk~environment variable.  In the event
\texttt{KADATH\_DIR} is not set, the build script will check \hk~ for a previously built library.
If one is found, this library will be used.  If one wishes to force building FUKA during the ETK
build procedure, they can set
\begin{verbatim}
    KADATH_DIR = BUILD
\end{verbatim}
in the ETK installation configuration.

\subsection*{Build}
If a previously built version of the FUKA library is not found, a manual installation is done during
the ETK build procedure.  The FUKA library is a git submodule of the KadathThorn.  Although the
directory is initialized when the thorn is cloned using GitComponents, the files do not exist.
When the build script decides that FUKA must be built from scratch, 
the script will first pull the FUKA distribution using git.  Afterwards, the files are stored in
a temporary build location to build the FUKA library.

It is important to note:
\begin{itemize}
    \item The elliptic solver and the initial data codes are not compiled in this procedure
    \item Once the build procedure is finished, only the static library and equations of state are retained
\end{itemize}

\section{Using This Thorn}
This thorn is not used other than to ensure the FUKA library is available for the KadathImporter thorn

If one utilises the equations of state stored in \hk/eos, the \hk~environment variable must be
correctly defined in the environment of the computer processing the initial data.

\subsection{Obtaining This Thorn}
This thorn can be obtained from \cite{fukaws}

\subsection{Support and Feedback}
The current active developer and maintainer is
\begin{verbatim}
    Samuel Tootle - tootle@itp.uni-frankfurt.de
\end{verbatim}

\section{History}
The pre-ETK inclusion KadathThorn and KadathImporter thorns were developed by L. Jens Papenfort, 
Samuel Tootle, and Elias Most for binary black hole and binary neutron star initial data. 
Samuel Tootle later rewrote the thorns to encapsulate all initial data available
from FUKA, updated to meet ETK inclusion standards, and wrote the documentation.

\subsection{Acknowledgements}
The authors are grateful to the continued support by Philippe Grandclement, author of
the KADATH spectral library, throughout the development of the FUKA codes.

\begin{thebibliography}{9}
    \bibitem{Papenfort2021a}
    "Papenfort, L. Jens et al", \emph{New public code for initial data of unequal-mass, spinning compact-object binaries},
        "10.1103/PhysRevD.104.024057"
    \bibitem{fukacodes}
    \emph{Frankfurt University/KADATH (codes)}, "{https://bitbucket.org/fukaws/fuka}"
    \bibitem{fukaws}
    \emph{Frankfurt University/KADATH (workspace)}, "{https://bitbucket.org/fukaws/}"

\end{thebibliography}

% Do not delete next line
% END CACTUS THORNGUIDE

\end{document}
